%%
%% This is file `sample-sigconf.tex',
%% generated with the docstrip utility.
%%
%% The original source files were:
%%
%% samples.dtx  (with options: `sigconf')
%% 
%% IMPORTANT NOTICE:
%% 
%% For the copyright see the source file.
%% 
%% Any modified versions of this file must be renamed
%% with new filenames distinct from sample-sigconf.tex.
%% 
%% For distribution of the original source see the terms
%% for copying and modification in the file samples.dtx.
%% 
%% This generated file may be distributed as long as the
%% original source files, as listed above, are part of the
%% same distribution. (The sources need not necessarily be
%% in the same archive or directory.)
%%
%%
%% Commands for TeXCount
%TC:macro \cite [option:text,text]
%TC:macro \citep [option:text,text]
%TC:macro \citet [option:text,text]
%TC:envir table 0 1
%TC:envir table* 0 1
%TC:envir tabular [ignore] word
%TC:envir displaymath 0 word
%TC:envir math 0 word
%TC:envir comment 0 0
%%
%%
%% The first command in your LaTeX source must be the \documentclass command.
\documentclass[sigconf]{acmart}

%%
%% \BibTeX command to typeset BibTeX logo in the docs
\AtBeginDocument{%
  \providecommand\BibTeX{{%
    \normalfont B\kern-0.5em{\scshape i\kern-0.25em b}\kern-0.8em\TeX}}}

%% Rights management information.  This information is sent to you
%% when you complete the rights form.  These commands have SAMPLE
%% values in them; it is your responsibility as an author to replace
%% the commands and values with those provided to you when you
%% complete the rights form.
% TODO
\setcopyright{acmcopyright}
\copyrightyear{2023}
\acmYear{2023}
\acmDOI{10.1145/1122445.1122456}

%% These commands are for a PROCEEDINGS abstract or paper.
% TODO:
\acmConference[GECCO 2023]{Genetic
and Evolutionary Computation Conference Companion}{2023}{Portugal}
% \acmConference[Woodstock '18]{Woodstock '18: ACM Symposium on Neural
%   Gaze Detection}{June 03--05, 2018}{Woodstock, NY}
% \acmBooktitle{Woodstock '18: ACM Symposium on Neural Gaze Detection,
%   June 03--05, 2018, Woodstock, NY}
% \acmPrice{15.00}
% \acmISBN{978-1-4503-XXXX-X/18/06}


%%
%% Submission ID.
%% Use this when submitting an article to a sponsored event. You'll
%% receive a unique submission ID from the organizers
%% of the event, and this ID should be used as the parameter to this command.
%%\acmSubmissionID{123-A56-BU3}

%%
%% The majority of ACM publications use numbered citations and
%% references.  The command \citestyle{authoryear} switches to the
%% "author year" style.
%%
%% If you are preparing content for an event
%% sponsored by ACM SIGGRAPH, you must use the "author year" style of
%% citations and references.
%% Uncommenting
%% the next command will enable that style.
%%\citestyle{acmauthoryear}

%%
%% end of the preamble, start of the body of the document source.
\begin{document}

%%
%% The "title" command has an optional parameter,
%% allowing the author to define a "short title" to be used in page headers.
\title{Calculating selection probabilities under lexicase selection is NP-Hard}

%%
%% The "author" command and its associated commands are used to define
%% the authors and their affiliations.
%% Of note is the shared affiliation of the first two authors, and the
%% "authornote" and "authornotemark" commands
%% used to denote shared contribution to the research.

\author{Emily Dolson}
 \email{dolsonem@msu.edu}
 \orcid{0000-0001-8616-4898}
 \affiliation{%
   \institution{Michigan State University}
   \city{East Lansing}
   \state{Michigan}
   \country{USA}
 }

%%
%% By default, the full list of authors will be used in the page
%% headers. Often, this list is too long, and will overlap
%% other information printed in the page headers. This command allows
%% the author to define a more concise list
%% of authors' names for this purpose.
\renewcommand{\shortauthors}{Shahbandegan and Dolson}

%%
%% The abstract is a short summary of the work to be presented in the
%% article.
\begin{abstract}
 % TODO
\end{abstract}

%%
%% The code below is generated by the tool at http://dl.acm.org/ccs.cfm.
%% Please copy and paste the code instead of the example below.
%%
% TODO:
% \begin{CCSXML}
% <ccs2012>
%  <concept>
%   <concept_id>10010520.10010553.10010562</concept_id>
%   <concept_desc>Computer systems organization~Embedded systems</concept_desc>
%   <concept_significance>500</concept_significance>
%  </concept>
%  <concept>
%   <concept_id>10010520.10010575.10010755</concept_id>
%   <concept_desc>Computer systems organization~Redundancy</concept_desc>
%   <concept_significance>300</concept_significance>
%  </concept>
%  <concept>
%   <concept_id>10010520.10010553.10010554</concept_id>
%   <concept_desc>Computer systems organization~Robotics</concept_desc>
%   <concept_significance>100</concept_significance>
%  </concept>
%  <concept>
%   <concept_id>10003033.10003083.10003095</concept_id>
%   <concept_desc>Networks~Network reliability</concept_desc>
%   <concept_significance>100</concept_significance>
%  </concept>
% </ccs2012>
% \end{CCSXML}

% \ccsdesc[500]{Computer systems organization~Embedded systems}
% \ccsdesc[300]{Computer systems organization~Redundancy}
% \ccsdesc{Computer systems organization~Robotics}
% \ccsdesc[100]{Networks~Network reliability}

%%
%% Keywords. The author(s) should pick words that accurately describe
%% the work being presented. Separate the keywords with commas.
\keywords{datasets, neural networks, gaze detection, text tagging}

%% A "teaser" image appears between the author and affiliation
%% information and the body of the document, and typically spans the
%% page.
% TODO (optionally)
% \begin{teaserfigure}
%   \includegraphics[width=\textwidth]{sampleteaser}
%   \caption{Seattle Mariners at Spring Training, 2010.}
%   \Description{Enjoying the baseball game from the third-base
%   seats. Ichiro Suzuki preparing to bat.}
%   \label{fig:teaser}
% \end{teaserfigure}

%%
%% This command processes the author and affiliation and title
%% information and builds the first part of the formatted document.
\maketitle

\section{Introduction}

% Lexicase selection is very good 
% It is tempting to try to calculate probabilities of selection under lexicase and use those
% It would also be handy to be able to calculate them for various theoretical purposes
% So it would be useful to know if it's possible to do so efficiently

% Turns out it's not
% Here we show that lexicase is NP-Hard by reduction from Satisfiability
% Also applies to epsilon lexicase because we can trivially reduce lexicase to epsilon lexicase

\section{Preliminaries}

\subsection{Definition of the Lexicase Selection Probabilities Problem}
% From looking at other papers in the theory track, it looks like it's common to include
% a section like this that clarifies assumptions/expectations etc.

\begin{definition}
Input: a vector, $P$, of $N$ vectors of length $M$. 

Output: a vector of length $N$ indicating the probability of each vector in $P$ being selected in a single round of lexicase selection.
\end{definition}

% Decision version

\subsection{Satisfiability ({\sc SAT})}

$v$ variables, $c$ clauses (note, we are using $v$ rather than the traditional $n$ to avoid confusion with the $n$ in lexicase)

\subsection{Brief review of computational complexity theory}

For readers unfamiliar with the $NP-Hard$ class of problems and how we prove that problems are in this class, we offer a capsule summary. 

In order to talk about how hard problems are, we place them into classes. Many problems that we regularly encounter are in the class $NP$ (non-deterministically polynomial), which is the set of problems for which we can verify in polynomial time that a given solution is correct. Some problems in $NP$ are ``easy'' in the sense that polynomial-time solutions for them have been found. For the hardest problems in $NP$, no polynomial-time solution has been found thus far and it is thought to be unlikely that one exists. $NP-Hard$ problems are the class of problems that are at least as hard as the hardest problems in $NP$. Thus, if a problem is $NP-Hard$, it is unlikely to be solvable in polynomial time. If calculating lexicase selection probabilities is $NP-Hard$, that is useful to know because it tells us that efforts to find a polynomial time algorithm to calculate them are unlikely to pay off.

To prove that a problem is $NP-Hard$, we must take a problem that we already know is $NP-Hard$ show that it reduces (in polynomial time) to the problem we're interested in.  Here, we will reduce {\sc SAT} to {\sc lex-prob}, i.e. we will convert instances of {\sc SAT} to instances of {\sc lex-prob} that, when solved, will also tell us the solution to the underlying {\sc SAT} instance. By showing that any instance of {\sc SAT} can be converted to an equivalent instance of {\sc lex-prob} in polynomial time, we will prove that {\sc lex-prob} is at least as hard as {\sc SAT}. If we came up with a polynomial time algorithm that solved {\sc lex-prob}, that would mean we also have a polynomial time algorithm for {\sc SAT} (and, indeed, all of the $NP-Complete$ problems). As the suspicion of most experts is that {\sc SAT} cannot be solved in polynomial time, this would mean that {\sc lex-prob} likely cannot be solved in polynomial time either.



\section{Calculating probabilities under lexicase selection is NP-Hard}

The lexicase selection probabilities problem is not in the set NP. There is no polynomial-sized certificate that could be returned that would allow someone to verify that the assigned probabilities are correct. Thus, the lexicase selection probabilities problem is not NP-Complete. However, we will show that it is NP-Hard.

We will show that . Or, more formally:

{\sc SAT} $\leq_{P}$ {\sc $\epsilon$-lex-prob-decision} $\leq_{P}$ {\sc lex-prob-decision} $\leq_{P}$ {\sc lex-prob}

\begin{theorem}
\label{lexicasetheorem}
The {\sc $\epsilon$-lex-prob} problem is NP-Hard.
\end{theorem}

We will prove this theorem by reducing Satisfiability ({\sc SAT}), a well-known NP-Complete problem [CITE Karp], to {\sc $\epsilon$-lex-prob} in polynomial time.

\begin{proof}
Given an instance of {\sc SAT} with $c$ clauses and $v$ variables, we can create an instance of {\sc $\epsilon$-lex-prob} with a population containing $N = 1 + c*v*2 + v$ vectors of size $M = c*v*2 + v*2$. $\epsilon$ will be set to .1. 

The first $c*v*2$ positions in each vector correspond to the clauses in the original {\sc SAT} instance. Each clause has $v*2$ positions, one for each variable and the negation of each variable. We will refer to these positions as $C_{CiXj}$, where $i$ is the index of the clause the position is part of and $j$ is the id of the variable the position refers to. For example, $C_{C2X3}$ would be the position corresponding to the value of the variable $X3$ in clause 2.

The next $v*2$ positions in each vector are the ones that will be used to explore different variable assignments. The order in which lexicase selection picks from these positions will determine which variable assignments are made. We will refer to the position corresponding to choosing to set $Xi$ to true as $D_{Xi}$ (whereas the position that corresponds to setting it to false would be $D_{!Xi}$).

The first vector in this population represents the {\sc SAT} instance. We will call this the focal vector. Its $C_{CiXj}$ values will be set to indicate which variables are in each clause. If clause $i$ contains variable $j$, then $C_{CiXj}$ in the focal vector is set to 1. Otherwise, it is set to 0. All $D_{Xi}$ values are set to .9.

The next $c*v*2$ vectors in the population correspond to possible variable assignments for each clause. We will call these the variable vectors. The $C_{CiXj}$ positions in these vectors are identical to the focal vector except in the clause corresponding to that variable vector. There, the position connected to the negation of the variable corresponding to the vector will be set to 2.

$D_{Xi}$ values for the variable vectors are set to .9, except in the position corresponding to that variable and its negation. For variable vectors corresponding to $Xi$, $D_{!Xi}$ will be set to .8, and $D_{Xi}$ will be set to 1. For variable vectors corresponding to $!Xi$, $D_{!Xi}$ will be set to 1, and $D_{Xi}$ will be set to .8.

Lastly, there are $v$ vectors that we will refer to as ``timing'' vectors. These vectors have all $C_{CiXj}$ positions set to 3. Each timing vector corresponds to a single variable and its negation. $D_{Xi}$ and $D_{!Xi}$ are set to 0 when $i$ is that vector's corresponding variable. Otherwise, they are set to .9.

The timing vectors insure that, before any of the $C_{CiXj}$ positions are selected, values have to be assigned to all variables by selecting either $D_{Xi}$ or $D_{!Xi}$ for all $i$. Any ordering of positions in which a $C_{CiXj}$ position is selected prematurely will result in one of the timing vectors winning.

When $D_{Xi}$ is selected, it eliminates all variable vectors corresponding to $D_{!Xi}$, and visa versa, because .8 is less than 1 by more than $\epsilon$. The other vectors, however, are not eliminated, as .9 is within $\epsilon$ of 1. Selecting $D_{Xi}$ or $D_{!Xi}$ also eliminates the timing vector corresponding to variable $i$, which has 0s in both positions. Selecting a variable after selection its negation or the other way around will have no effect, as all remaining values for that position will be .8 or .9 and so will be within $\epsilon$ of each other and tie.

Once variable assignments have been made, $C_{CiXj}$s can now safely be selected. The focal vector will only have a chance of winning if, for each clause, there is at least one position where the focal vector has a 1 and no variable vectors with 2s remain in contention.

% Wait, do we need .9s instead of 1s in th $D_{Xi}$? I think we do

\end{proof}

\section{Software???}

\section{Discussion}

% Talk about the implications
% Give equation for what population size you need relative to number of objectives to ensure
% that being best at a single objective is enough to survive permenantly

% Go into details of using that to choose parameters etc.
% (depending on how many objective you have, how opposed they each are, etc.)

\section{Conclusion}

%%
%% The acknowledgments section is defined using the "acks" environment
%% (and NOT an unnumbered section). This ensures the proper
%% identification of the section in the article metadata, and the
%% consistent spelling of the heading.
\begin{acks}
%TODO
We would like to thank the ECODE lab
\end{acks}

%%
%% The next two lines define the bibliography style to be used, and
%% the bibliography file.
\bibliographystyle{ACM-Reference-Format}
\bibliography{sample-base}

%%
%% If your work has an appendix, this is the place to put it.
% \appendix

% \section{Research Methods}

% \subsection{Part One}

% Lorem ipsum dolor sit amet, consectetur adipiscing elit. Morbi
% malesuada, quam in pulvinar varius, metus nunc fermentum urna, id
% sollicitudin purus odio sit amet enim. Aliquam ullamcorper eu ipsum
% vel mollis. Curabitur quis dictum nisl. Phasellus vel semper risus, et
% lacinia dolor. Integer ultricies commodo sem nec semper.

% \subsection{Part Two}

% Etiam commodo feugiat nisl pulvinar pellentesque. Etiam auctor sodales
% ligula, non varius nibh pulvinar semper. Suspendisse nec lectus non
% ipsum convallis congue hendrerit vitae sapien. Donec at laoreet
% eros. Vivamus non purus placerat, scelerisque diam eu, cursus
% ante. Etiam aliquam tortor auctor efficitur mattis.

% \section{Online Resources}

% Nam id fermentum dui. Suspendisse sagittis tortor a nulla mollis, in
% pulvinar ex pretium. Sed interdum orci quis metus euismod, et sagittis
% enim maximus. Vestibulum gravida massa ut felis suscipit
% congue. Quisque mattis elit a risus ultrices commodo venenatis eget
% dui. Etiam sagittis eleifend elementum.

% Nam interdum magna at lectus dignissim, ac dignissim lorem
% rhoncus. Maecenas eu arcu ac neque placerat aliquam. Nunc pulvinar
% massa et mattis lacinia.

\end{document}
\endinput
%%
%% End of file `sample-sigconf.tex'.
